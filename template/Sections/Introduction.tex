This project aims to solve the task presented at PAN, Profiling Irony and Stereotype Spreaders on Twitter (IROSTEREO) 2022\footnote{https://pan.webis.de/clef22/pan22-web/author-profiling.html}, which consists of classifying authors as irony and stereotype spreaders given a set of English tweets. Because irony is employed ``to mean the opposite to what is literally stated", per the task authors, it can be used to scorn or stereotype vulnerable groups in ways that avoid conventional moderation techniques \cite{greene2019deplorable}. Identifying irony and stereotype spreaders could help improve identification of hate speech and cyberbullying for moderation purposes \cite{cyberbullying, waseem2016hateful} and contribute to the problem of disambiguation in natural language processing \cite{reyes2012humor, sarcasm_detection}. 

